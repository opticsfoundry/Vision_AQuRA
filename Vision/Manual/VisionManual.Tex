\documentclass[10pt]{article}
\usepackage{epsfig}
\usepackage[english]{babel}
\usepackage{latexsym}
\textwidth15.5cm \textheight21.5cm \oddsidemargin2mm
\evensidemargin2mm
\parskip 1mm
\setlength{\parindent}{0.5cm}

\begin{document}

{\huge{Manual of "Vision"}}

\section{Introduction}

Vision is a data acquisition and treatment program for cold atom
physics experiments, especially Bose-Einstein condensation (BEC)
experiments. It also acts as a sort of automatic lab book keeping
protocols of every experimental run executed. It is used on at
least five BEC experiments worldwide and was originally developed
for the Lithium experiment at the E.N.S. in Paris. The general
setup of the computer system used with a BEC experiment using
Vision is the following. A control computer, running for example
the program "Control" (see
\begin{verbatim}http://george.ph.utexas.edu/~control/index.html\end{verbatim})
generates the analog and digital waveforms used to steer the
experiment. This control computer is linked via ethernet with the
data acquisition computer running "Vision". The control computer
sends commands to prepare the camera to take images to "Vision"
via TCP/IP or the serial port. Vision communicates with the camera
and acquires the images. After an experimental run, "Control"
transmits the parameters of the experimental sequence to "Vision"
which stores them together with the acquired images. Vision treats
the raw images and obtains absorption or fluorescence pictures
from which it obtains measurements like the atom number or the
size of the atomic cloud. You can use any other control system
with Vision as long as you can implement the simple communications
protocol between Vision and your control program over TCP/IP or
the serial port. When you adapt Vision to your system you have to
implement support for your camera system. Vision comes built in
with support for some Princeton, Andor and Apogee cameras. Some
cameras are delivered with drivers which do not work with Borland
C++ 5.02 in which Vision is written. For those cameras, eg. the
Apogee Alta series, a small helper program supports Vision. It is
called "ApogeeServer" and can be downloaded on the webpage given
above. It is written in Visual C++ and can thus be adapted to all
cameras delivered with drivers compatible with this language.
Vision and ApogeeServer communicate over ethernet using TCP/IP to
exchange camera commands and picture data.

This document describes the user interface of Vision and its most
important features. Then it gives guidelines how to adapt it to
your system and implement your own camera driver. Finally the
communication protocol between the control computer and Vision is
described. With this information it should be easy to adapt Vision
to your system.

\section{User interface}

Visions user interface is divided in several sections. The most
important one is the big sub window in which normally the
absorption picture of the last data point under consideration is
prominently displayed. To the left and below you see profiles of
the picture displayed. In these profiles you can see blue and red
bars which you can position by using the left and right buttons of
the mouse. The rectangular area marked in blue marks the position
of the atoms. The area marked in red, but excluding the blue area
gives Vision a region where no atoms are present from which it can
determine an eventual background offset. The profiles shown are
usually the sum of the lines or columns in between the blue lines.
Gauss fits are automatically performed to the profiles in the
region marked by the red lines. The atom number is determined by
integrating the signal in the blue marked area and subtracting an
eventual offset obtained from the red region.

The result of the atom number calculation and the Gauss fits are
displayed in the window "Gauss and integral fit result". "Dens" is
the peak optical density, "X0" and "Y0" give the position of the
center of the gauss fits in millimeters. "sx" and "sy" are the
sigma widths of the Gaussians. "Ox" and "Oy" is the value of the
offset of the Gaussians. "NG" is the atom number extracted from
the Gaussian fit and "NI" the one obtained from integration over
the image in the blue marked region. "n0" is the peak density in
cm$^{-3}$. "Fl" the fluorescence count of the MOT just prior to
starting the experiment. "NOf" is the offset in optical density
extracted from the red region excluding the blue region. "ReFl" is
user specific and can be for example the fluorescence after
recapture in a MOT from an optical or magnetic trap. "TF/TC" is
the ratio between the Fermi and the critical temperature of the
gas. "Rx" is the Thomas Fermi radius.

To obtain the latter values, "Control" did transmit the trap
parameters used to Vision. They are displayed in the box
"Measurement parameters". "Camera" is the number of the camera
used for this image. "Det" is the detuning of the absorption beam.
"TOF" is the expansion time (time of flight). "Isotop" is the
Isotop used for example Lithium 6 or Lithium 7. This was used in
the original Lithium experiment for which Vision was created. It
can mean completely different things in your experiment for
example 6 might mean Potassium and 7 Rubidium or 6 might mean
vertical imaging system and 7 horizontal imaging system. Trap is a
number describing the type of trap. For trap type 0 = Ioffe
Pritchard trap, the next three parameters "B", "G" and "C" give
the bias in Gauss, the gradient in Gauss/cm and the curvature in
Gauss/cm$^2$. "P1" to "P4" are user defined parameters. If you are
taking many experimental points in a series of measurements, those
are the up to four parameters which can be varied from point to
point. Usually only one parameter is varied, eg. the expansion
time and then displayed under "P1". The last line displays a text
string describing the experimental run. Each stage of the
experimental sequence is given a specific letter and all stages
used to obtain a certain data point are chained next to each other
giving the shortcut string displayed here. At the end of the
string the final trap type is written out in normal words. This
descriptive string is generated by "Control" and send over to
Vision.

Using the parameters from these two boxes Vision calculates more
advanced things and displays them in the box "Temperature and Atom
number fit results". Which parameters here are valid for a given
data point and how they are calculated depends on the context in
which the picture lays and on fits performed on a series of
experimental runs. This will be explained later.

In the main absorption picture window you also see a cross cursor.
It usually marks the position of maximum optical density in the
blue region, unless it has been fixed to stay at a certain
position using the menu option "Options->Fix cursor". If you move
the cursor around with the mouse, the profiles will be updated and
display not any more the sum over the image but the local one
pixel line cut at the position of the cross. The position of the
cursor is displayed in the box "Cursor Position" under "X" and
"Y". "H" is the optical density (height) of the image at the
center of the cross. "Xp" is the position of the left blue bar and
"Dx" the distance between the horizontal blue bars in millimeters.
"Yp" and "Dy" are defined accordingly for the vertical blue bars.

To the right of the main picture the color scale is displayed. You
can change it by clicking and moving its upper or lower border
with the left button. The right mouse button will display white
iso optical density lines.

To the right you find six small windows. The window "Fluorescence"
is used for test images taken by using the menu option
"Camera->... picture" or "Camera->... film". For absorption images
"Raw absorption" displays the probe beam profile with atoms
present, "Probe" displays the probe beam profile without atoms,
and "Noise" displays the noise present without probe beam from
scattered light. The "Absorption" picture is ("Raw
absorption"-"Noise")/("Probe"-"Noise"). The optical density
picture is -$\ln$("Absorption")/$\sigma_{abs}$ and thus after the
absorption law equal to the 1D integral over the atomic density
along the absorption beam. That is why the 2D integral over the
"optical density" picture is equal to the 3D integral over the
atomic density distribution and thus the atom number. You have the
responsibility to enter the correct Clebsch-Gordan coefficient of
your atomic transition in the file "setup.h" of Vision so that
Vision can calculate $\sigma_{abs}$. Clicking on any of the images
loads it to the main picture window and starts the extraction of
the parameters from it.

For fluorescence images "Raw absorption" is the fluorescence image
with atoms and "Probe" is the background image without atoms.
"Absorption" and "Optical density" are both proportional to "Raw
absorption"-"Probe". "Optical density" is re scaled so that the
minimum and maximum signal just span the whole color scale,
whereas "Absorption" is scaled using a value for the maximum
absorption transmitted to Vision by "Control".

The twelve images to the right display the main image of the last
six experimental runs. You can use the left and right column as
you want, eg. for vertical and horizontal cameras or for different
isotopes. The picture actually displayed here and in the main
window is selected with the menu option "Show->...". The last
acquired or loaded image is marked by ">". The filename of the
data point is marked below the picture. Clicking on any of those
pictures loads the corresponding data point to the main picture.

Below the six raw images a white region displays the result of the
measurement done over the parameter varied during the measurement.
The parameter(s) displayed is selected with the checkboxes to the
left of the frame. The checkboxes on top select if only data
belonging to vertical or horizontal images or both is displayed.
These checkboxes also determine which image should be the one
displayed after acquisition in the main window. Below the graph
the name of the parameter varied is displayed. That's usually the
picture number. But in a measurement series it is one of the
parameters varied, eg. the expansion time. A data point is loaded
when you click on it and you can load the next or the previous
data point with the two arrow buttons at the lower right side of
the box. You can mark a point as invalid by clicking on it with
the right mouse button. A point is deleted from a series by
loading the point by clicking on it and the using the menu option
"Measurement->Delete measurements". A single data point is loaded
with "Measurements->Load measurement".

The menu "Series" gives you more options to do with measurement
series. "Load" loads a series from hard disk, CD or DVD. "Add"
adds another series to the ones loaded already. "Save" saves the
points loaded. "Shift parameters" toggles to the next parameter
varied for series during which multiple parameters have been
changed. "Sort series" sorts the series after the value of the
varied parameter. This is useful since during measurements usually
the order of the parameter values measured is randomized to fight
experimental drifts. "New series", "Start series" and "Stop
series" enables you to create series manually, which means that
data points are stored in their proper subdirectory on hard disk.

Simple fits can be performed to the data displayed using the "Fit"
menu. But it is recommended to load the ASCII files describing the
experimental series with a program like "Origin" and performing
the fits there.

Below the data point window the lab book window is displayed. It
contains a list with a description of all data points measured
that day containing the name of the data point and the names and
values changed since the last data point. When loading a new data
point this list automatically scrolls to the description of this
point if possible. Here an example for the description of a
datapoint:
\begin{verbatim}
 *P-0011A7 MOMP Magnetic trap
  ExpansionTime=8.000 (5.000)
\end{verbatim}
"P-0011A7" is the filename and tells you that it is the eleventh
point of its series taken with the horizontal imaging system (or
Lithium 7, or whatever you defined 7 to signify). The atoms come
from a MOT "M", underwent optical pumping "O", where captured in a
magnetic trap "M" and an absorption picture was taken of them "P".
The last stage before picture taking was a Magnetic trap. The only
parameter varied since the last image was the "ExpansionTime" and
it went from 5\,ms to 8\,ms.

Below this window you have buttons to steer 2D fits and 2D fit
results. This subject is a bit involved and not often used anyway
so I won't bother to describe it further unless somebody asks me
for it.

At the very bottom of the window you can see a status bar,
displaying the messages exchanged between Control and Vision or
fit results.

We have discussed already most of the menu options. Here are the
remaining ones. In the "File" menu you will find export options
for the image displayed in the main window. "Save as matrix" saves
the image as an ASCII table, the others as BMP or TGA images.
"Save low byte as BMP" saves only the lowest 8bit of the image and
can be used to measure the camera noise. "Save Profile" saves
ASCII tables of the profiles.

"Calc" contains commands to execute on all pictures on the series
which is loaded at the moment. The only important ones are
"ReCalc" which recalculates the fits of each image and "Create
film" which creates a subfolder into which BMP files of each main
window frame are exported with a filename that makes importing to
a program like "GIF construction set professional" easy to create
GIF animations.

There are various fits in the "Fit" menu, of which the most
important one is "Fit temperature" which obtains the temperature
of a cloud from its expansion behavior.

The most important options in the "Options" menu are the "Fix
cursor" and "Zoom" commands. Sometimes it can be useful to alter
the standard behavior of the gaussian fits to the profiles by
allowing a slope or disabling the fit offset which can be done
under "Options->1D Fit options".

\section{File system}

The main directory for data storage is declared in "setup.h". This
can be "C:\\Rubidium\\Data". Here you find "Logbook.dat"
describing all experimental series ever executed with your system.
Also reference images for the probe beam profile and the noise
images are stored here, in case you chose the option to take only
the image with atoms to speed up data acquisition. The file
"Message.txt" contains the last message send to Vision by Control
and can be looked up over the LAN to see where Vision is. You can
also connect a remote version of Vision over LAN to the master
Vision and you get a duplicate of the Vision screen contents on a
remote computer in your LAN. Vision creates a new subfolder for
each day, eg. "C:\\Rubidium\\Data\\04May07". In this subfolder
"DataP-7.dat" and "DataP-6.dat" contain ASCII tables describing
the measurements you did. Load these files with a program like
"Origin" to fit or print your data. "Logbook.txt" contains the
logbook displayed in the logbook window of Vision. Each data point
consists of several files. "P-0001\_6.txt" and "P-0001\_7.txt"
contain all the parameters used for the data point (here data
point number 1). Those files are very useful. For example if you
want to reproduce a data point but something doesn't work and you
don't know which one of the hundreds of parameters describing your
experimental sequence has changed, use a program like "CSDiff" to
compare "P-xxxx\_6.txt" with "P-yyyy\_6.txt" and you will find the
difference instantly. Next a series of TGA files like
"P-0001A6.TGA" contain the raw 16bit pictures. "P-0001A7.bmp" and
"P-0001A6.bmp" contain 8bit bitmaps of the main window picture,
normally the optical density picture. Put Windows Explorer in
thumbnail display mode and you have a nice overview of all
pictures you did a certain day. For each measurement series a
subdirectory like "S1" or "S2" and so on is created and the same
types of files are stored here. For the creation of films
directories "F1" and so on are created.

\section{Setup of Vision}

You need to put windows in a 32bit color screen mode. Next you
need to modify the parameters in "setup.h" for your system. This
includes selecting the mass and Clebsch-Gordan coefficient of your
atom, selecting your camera system, configuring the directories in
which you want to store your acquired data and eventually the
TCP/IP address of the computer running "ApogeeServer".

The file "Vision.cfg" stored in the main directory contains setup
information mainly used for connection of a second version of
Vision over LAN to Vision on your experiment. This is not really
important and should just look like this by default:
\begin{verbatim}
DataDisk=2
DataDirectory=c:\Rubidium\Data
MainDirectory=c:\Rubidium
HardwareAccess=1
LiveDirectory=F
LiveDisk=5
LiveConnectTime=10000
\end{verbatim}

to be continued...

\section{Adding a new camera driver}

Essentially you look which camera driver that already exists is
nearest to the one you want to use and then you modify the files
corresponding to this driver. Say you have a slow scan camera for
which you have Borland C++ compatible drivers, then you might want
to just change the file "Andor.cpp" and let your camera appear to
the rest of Vision like the Andor camera. If you do not have
Borland C++ compatible drivers, but Visual C++ compatible drivers,
you need to use "ApogeeServer" and modify it for your camera using
the manual provided with "ApogeeServer". To the rest of Vision
your camera appears as the NetAlta camera.

to be continued...

\section{Communication with Control}

It is easiest just to use "Control" to control your setup. If you
do not want to do that, then download "Control" all the same from
my webpage and look at the CVision class. It contains all commands
necessary to communicate with vision. Look at the CSequence class
and search for all calls of CVision to understand how Control
talks to Vision. Essentially before each image is taken a command
"VisionSetCameraParameters" is send setting up the camera and then
a "VisionTakeAbsorptionPicture" command starts the image
acquisition sequence. In addition "VisionStartSeries" and
"VisionStopSeries" commands can start a new measurement series.

to be continued...

Here a protocol of communication between "Control" and "Vision" to
take one image. Messages marked with ">>" are send from "Control"
to "Vision". "<<" marks the other way round. All messages start
with "*" and end with "\#" even when not marked as such in this
example. "//" marks comments which do not actually appear during
the communication

\begin{verbatim}
 // check if Vision is there
 >> *VisionReady#
 << Ready
 // check if camera is ready
 >> *VisionCameraReady#
 // Vision tells us it is ready to receive parameters:
 << Ready
 // Number of camera:
 >> *       8#
 // ok, camera is ready:
 << VisionCameraReady
 // Main picture display mode
 >> *VisionMainPictureOpticalDensity#
 << Ready
 // Set the camera parameters:
 // Look up Visiondg.cpp and search for VisionSetCameraParameters
 // to understand their meaning, or look up CVision.cpp in
 // "Control"
 >> *VisionSetCameraParameters#
 << Ready
 >> *       8#
 >> * 0.00300#
 >> * 0.00300#
 >> *       0#
 >> *       0#
 >> *    1024#
 >> *    1024#
 >> *       1#
 >> *       1#
 >> *20.00000#
 >> *-30.00000#
 // Now take the picture:
 >> *VisionTakeAbsorptionPicture#
 << Ready
 >> *       0#
 >> *       8#
 >> *P#
 >> *-#
 >> *       6#
 >> *       0#
 >> *       0#
 >> *16384.00000#
 >> * 2.00000#
 >> * 2.00000#
 >> * 0.00000#
 >> * 0.00000#
 >> * 5.00000#
 >> * 5.00000#
 >> * 1.00000#
 >> *1364.57600#
 >> * 0.00000#
 >> *       0#
 >> *       2#
 >> *MoOMCP Magnetic trap#
 << Ready
 << Ready
 // Picture has been taken, now send parameters
 >> *VisionSynchronizeParameters#
 << Ready
 // since this is first image, send all
 // parameters, later send only changed ones
 << SendAll
 // a double parameter marked with D
 >> *DVerticalSheetPositionMax=45.000#
 >> *DVerticalSheetPositionConversion=0.763#
 >> *DVerticalSheetCenterFrequency=100.000#
 >> *DHorizontalSheetPositionMax=50.000#
 .
 .
 .
 >> *DMOTEarthFieldCompensationCurrentZ=0.315#
 >> *DEarthFieldCompensationCurrentY=1.500#
 >> *DEarthFieldCompensationCurrentX=0.820#
 >> *DMOTAOMIntensity=0.000#
 // a bool parameter marked with B
 >> *BUseVacuumChamberShutter=FALSE#
 >> *BStartWithTinyMOT=FALSE#
 // a title marked with T
 >> *T*** MOT,CMOT, Molasses parameters ***=#
 // These where all the parameters
 >> *VisionSynchronizeParametersEnd#
 << Ready
 // Now send some data measured by control
 // during this experimental run like MOT
 // Fluorescence. Look up "Control"
 // CVision::SendPictureData
 // to understand meaning.
 >> * 0.20000#
 >> * 0.00000#
 >> * 0.00000#
 >> *28.00000#
 >> * 0.00000#
 >> * 0.00000#
 >> * 0.00000#
 >> *186.48000#
 >> * 0.00000#
 >> *2000.00000#
 >> * 0.00000#
 >> * 0.00000#
 >> * 0.00000#
 >> *      10#
 >> * 0.00000#
 >> *72790.76500#
 >> * 0.00000#
 >> * 0.00000#
 >> * 1.00000#
 >> * 0.00000#
 >> * 0.00000#
 >> * 0.00000#
 >> * 0.00000#
 >> * 0.00000#
\end{verbatim}

\end{document}
