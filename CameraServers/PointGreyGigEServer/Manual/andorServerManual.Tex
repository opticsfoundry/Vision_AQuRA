\documentclass[10pt]{article}
\usepackage{epsfig}
\usepackage[english]{babel}
\usepackage{latexsym}
\textwidth15.5cm \textheight21.5cm \oddsidemargin2mm
\evensidemargin2mm
\parskip 1mm
\setlength{\parindent}{0.5cm}

\begin{document}

\title{Manual of "AndorServer"}

\section{Introduction}

The Andor Alta camera drivers are ActiveX drivers. It turned out
to be hard to make them work under Borland C++. The data
acquisition program Vision is written in Borland C++ and needs
access to the Alta camera. The solution to this problem is
AndorServer. It is a small Visual C++ program which communicates
on one side with the Alta camera using the ActiveX drivers
delivered with the camera and on the other side with Vision over
ethernet using TCP/IP. The program can easily be modified to
support other camera systems than the Andor Alta and make these
cameras accessible to Vision as long as Visual C++ drivers are
delivered with the camera.

\section{Installation of the Alta system}

The Alta comes delivered with drivers. Call the install batch file
delivered with them. Sometimes the batch file doesn't work
correctly and you have to manually do what it intends to do:
copying the DLL and registering it with regsvr32. Power the Alta
up. Wait until the LEDs blink! Hook the Alta up to your system.
Windows asks for some drivers to be installed. Use Maxim, which is
delivered with the Alta to test if it works. Eventually switching
the Alta off and on, rebooting and verifying that the DLL and the
.SYS file are at the correct place and registered should make
everything work.

Probably you need to fabricate an external trigger cable. It is a
8pin mini din mal plug, like the Apples serial port connector. The
external trigger has to be hooked up on pin 1 and its ground to
pin 8.

\begin{verbatim}
 I/O for Alta
 ----------------------------
 Pin 1 - Trigger Input
 Pin 2 - Shutter Output
 Pin 3 - ShutterStrobe Output
 Pin 4 - External Shutter Input
 Pin 5 - External Readout Input
 Pin 6 - Timer Pulse Input
 Pin 7 is 12V from the camera head input
 Pin 8 is ground.

  Connector pin layout:

       1 2
      3 4   5
       6 7 8

  There is a bigger gap between
  pin 4 and 5 than anywhere else.
  Below pin 7 and to the sides of
  pin 1 and 2 are mechanical markers
  helping to orient the plug correctly
\end{verbatim}

\section{User interface}

The user interface displays the system status in a textbox to the
left and has menu options on buttons to the right. The "Expose"
button starts and internally triggered exposure. The "IO setup",
"LED setup", "Temperature" and "Search" buttons call the drivers
corresponding dialog boxes. The "Test trigger" button switches the
trigger line to be an output and toggles it a few times slowly
from low to high to test the external trigger cable. The "Expose
Ext" button starts an externally triggered exposure". The "Debug
start" button starts registering the TCP/IP communication in the
file "c:\\Debug.dat". "Debug stop" stops this. "Exit" quits the
program.

\section{Communication with Vision}

Here is a transcript of the TCP/IP communication with Vision. The
first column is the time of communication in milliseconds, the
second the direction ("<<" means from Vision to Andorserver) and
the third is the text actually send. In fact each text send starts
with "*" and ends with "\#", but these markers have been left out
ini this protocol.

\begin{verbatim}
 5902296 << Parameters
 5902437 << 0002
 5902437 << 0002
 5902437 << 0004
 5902437 << 1019
 5902437 << 0004
 5902437 << 1019
 5902437 << 0020
 5902437 << -030
 5902437 << 0001
 5902437 << 0000
 5903843 << Image
 5903968 << 0003
 5914484 >> Ready
 5914484 << Ready
 5914500 >> 516128
 5914500 >> SendData 516128
 5914718 << Ready
 5914718 >> 516128
 5914718 >> SendData 516128
 5914937 << Ready
 5914937 >> 516128
 5914953 >> SendData 516128
\end{verbatim}

The first command "Parameters" tells Andorserver that the image
parameters will be send. They follow next in this order: BinningX,
BinningY, XMin, XMax, YMin, YMax, exposure time in milliseconds,
goal temperature of CCD, external trigger (1=external,
0=internal), and the wait time between images in a series of
images to be taken.

The next command "Image" tells Andorserver to take Images. The
number of images to be taken is send as parameter. Andorserver
answers with "Ready". Vision tells Andorserver with "Ready" that
it is ready to receive data. Andorserver then send the number of
bytes in the picture and then the data of the pixture.

\section{Classes}

The application is initialized in CAndorServer::InitInstance().
If you want to change the IP port (normally 703) you do it here.
The camera is represented by CAndor and initialized with
CAndor::InitCamera(). Here you can modify the startup behavior
eg. define what the LEDs do or switch off the fans. The user
dialog is CAndorServerDlg and the TCP/IP interface is
CTCPIPServer.

\section{Events}

Two sorts of events can trigger something to happen. Either the
user pushes a button. For example the "Expose" button calls
CAndorServerDlg::OnExpose(). This method then calls the
appropriate functions of CAndor. The other is that Vision sends a
command. They are received in CTCPIPServer::ProcessMessage(). In
dependance of the message functions of CTCPIPServer executing the
message are called. You can add new commands here if you need to.
Look how CTCPIPServer::SetParameters or TakeImage communicate with
Vision and CAndor to learn how to add new functionality.
Essentially you read parameters from Vision using the ReadInt
function and you pass those parameters on to CAndor.

\section{Special remarques}

Our Alta camera driver has a problem accepting 0 as value for
yMin. It always converts it to 1. Combined with binning this makes
it necessary for us to start not with line 0 but line 4. The Alta
driver also sometimes stalled after an image with binning. That's
what the ResetSystem() commands are for. They avoid this situation
to occur.


Andor Camera pin layout:

pin 13: external trigger input


\end{document}
